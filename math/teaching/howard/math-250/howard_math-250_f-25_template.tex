\documentclass[a4paper,10pt,leqno]{article}

%------------------------------------------------------------

%	avoid modifying

\usepackage[T1]{fontenc}
\usepackage[utf8]{inputenc}
\usepackage{xcolor}
\usepackage{amsfonts, amsmath, amssymb, amsthm}
\usepackage{mathtools, microtype}
\usepackage{hyperref}
	% https://tex.stackexchange.com/a/142276
	\pdfstringdefDisableCommands{\let\enspace\empty\let\noindent\empty}
	\hypersetup{colorlinks}
\usepackage{lmodern}

% increases the line spacing
\linespread{1.2}

%------------------------------------------------------------

%	modify with caution

\numberwithin{equation}{section}
\theoremstyle{plain}
	\newtheorem{thm}{Theorem}[section]
	\newtheorem{lem}[thm]{Lemma}
\theoremstyle{definition}
	\newtheorem{df}[thm]{Definition}
\theoremstyle{remark}
	\newtheorem{rem}[thm]{Remark}

%------------------------------------------------------------

%	custom commands ("macros") could go below:



%------------------------------------------------------------

\begin{document}

\title{Title}
\author{Author}
\date{\today} % you can also enter the date manually here
\maketitle

\thispagestyle{empty}

%	\mainmatter

\section{Introduction}

\LaTeX{}, pronounced \emph{LAY-tek} or \emph{LAH-tek}, is a software system for typesetting documents.
It was created in the 1980s by the computer scientist Leslie Lamport, building on a more basic system called \TeX{}, which had been created in the 1970s by the computer scientist Donald Knuth.
Nowadays, these systems are extremely standard in scientific writing.

To create \LaTeX{} documents on your own computer, you need a \emph{compiler}, and for maximum convenience, an \emph{editor} as well.
You edit a \texttt{.tex} file in the editor, then click ``compile'' (or use a keyboard shortcut) to tell the compiler to create a \texttt{.pdf} document.
In doing so, the compiler will create a bunch of extra files with extensions like \texttt{.aux}, \texttt{.log}, and so on.
For this reason, it's helpful to keep all the documents for a given project within a single folder.

Alternatively, Overleaf (\url{https://www.overleaf.com/}) is a website where you can create \LaTeX{} documents online, after first creating a free account.
The makers also wrote a very nice guide to \LaTeX{} itself:
\begin{quote}  \url{https://www.overleaf.com/learn/latex/Learn_LaTeX_in_30_minutes}
\end{quote}
This webpage contains guidance on choosing and installing a compiler:
\begin{quote}
\url{https://www.overleaf.com/learn/latex/Choosing_a_LaTeX_Compiler}
\end{quote}

\section{Section}

\subsection{Subsection}

A \LaTeX{} document consists of
\begin{enumerate} 
\item 	a \verb|\documentclass[ ]{ }| command, specifying the type of document; 

\item 	a \emph{preamble} where packages can be loaded and custom commands/environments defined; and

\item 	a \texttt{document} environment, where the actual document is written.

\end{enumerate}
In general, the syntax for a \texttt{blah} environment is
\begin{quote}
\verb|\begin{blah}|\\
\verb|...|\\
\verb|\end{blah}|
\end{quote}
For instance, the list above was created using the \texttt{enumerate} environment.
To get an unnumbered list, use the \texttt{itemize} environment.
The hyperlinks in the previous section were typeset using the \verb|\url{ }| command within a \texttt{quote} environment.
If you want to learn how to make custom commands, see:
\begin{quote}
\url{https://www.overleaf.com/learn/latex/Commands#Defining_a_new_command}
\end{quote}

\subsection{Non-Mathematical Text}

To typeset quotation marks, use \texttt{\`{}} and \texttt{'}:
\begin{center}
\texttt{\`{}\`{}word'{}'}
\quad{gives}\quad
``word''.
\end{center}
Using \texttt{"} will not produce the correct double-quotation marks.

Use \texttt{\textbackslash{}cite[ ]\{ \}} to create a citation.
Use an en dash (in \TeX: \texttt{-{}-}) instead of a hyphen (\texttt{-}) to typeset page ranges.
For instance,
\begin{center}
\verb|\cite[82--83]{munkres}|
\quad{gives}\quad
\cite[82--83]{munkres}.
\end{center}
The hyperlink goes to a bibliography entry at the end of this document.

You can use commands like \verb|\textbf{ }| and  \verb|\textit{ }| to produce text in \textbf{bold} or \textit{italics}.
A shorter command for the latter is \verb|\emph{ }|.

\subsection{Mathematical Text}

To typeset math inline with non-math text (\emph{inline mode}), use \verb|$...$|.
For instance,
\begin{center}
\verb|the identity $a^2 + b^2 = c^2$|
\quad{gives}\quad
the identity $a^2 + b^2 = c^2$.
\end{center}
To typeset math as a centered display (\emph{display mode}), there are several methods.
The quickest is to use \verb|$$...$$|.
The \texttt{equation} environment will give the same result, but with a numbered label next to the display.
To omit the numbered label, use \texttt{equation*}.

I tend to use the \texttt{align} and \texttt{align*} environments for everything, because they let you line up expressions using \verb|&|:
\begin{align}\label{eq:1}
X 
	&= Y \cap \bigcup_{i = 1}^\infty Z_i\\
	&= \bigcup_{i = 1}^\infty {(Y \cap Z_i)}.
\end{align}
To make a multi-line display with a single label, put a \texttt{split} environment inside an \texttt{align} environment.

Note that \verb|\bigcap|, \verb|\bigcup| produce the large symbols $\bigcap, \bigcup$, whereas \verb|\cap|, \verb|\cup| produce the small symbols $\cap, \cup$.
The display above shows how the larger symbols have a different use from the smaller ones.

Completely separately, some commands have different inline and display appearances.
For instance, compare $C_n = \frac{1}{n + 1} \binom{2n}{n}$ to
\begin{align*}
C_n = \frac{1}{n + 1} \binom{2n}{n}.
\end{align*}
Use \verb|\displaystyle| and \verb|\textstyle| to modify this behavior.

Lastly, \LaTeX{} offers several different alphabets in math mode, including
\begin{itemize} 
\item 	\verb|\mathbb{ }| for blackboard boldface ($\mathbb{A}$),

\item 	\verb|\mathbf{ }| for ordinary boldface ($\mathbf{A}$), 

\item 	\verb|\mathcal{ }| for calligraphic ($\mathcal{A}$),

\end{itemize}
among others.
\LaTeX{} also offers several ways to decorate a symbol in math mode, including \verb|\bar{ }|, \verb|\hat{ }|, \verb|\widehat{ }|, \verb|\tilde{ }|, and \verb|\vec{ }|.

\subsection{More Math Environments} 

\begin{proof} 
The \texttt{proof} environment is used for proofs.
\end{proof} 

The \texttt{theoremstyle} commands in the preamble of this document define some other useful environments.

\begin{thm}\label{thm:1}
A theorem environment.
\end{thm}

\begin{lem}[Munkres]
You can add a label in parentheses, like the one here, to most environments.
\end{lem}

\begin{df}
A definition environment.
It has a link to Theorem \ref{thm:1}.
\end{df}

\begin{rem}
A remark environment.
It has a link to equation \eqref{eq:1}.
\end{rem}

%	\backmatter
\frenchspacing
\begin{thebibliography}{11}
\bibitem[M]{munkres}
	J. Munkres.
	\textsl{Topology}.
	2nd Edition.
	Pearson Education, Ltd.\ (2014).
	
%	keep a blank space above \end{thebibliography}, in this TeX file
	
\end{thebibliography}

\end{document}
