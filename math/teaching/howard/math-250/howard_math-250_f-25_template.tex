\documentclass[a4paper,10pt,leqno]{article}

%------------------------------------------------------------

%	avoid modifying

\usepackage[T1]{fontenc}
\usepackage[utf8]{inputenc}
\usepackage{xcolor}
\usepackage{amsfonts, amsmath, amssymb, amsthm}
\usepackage{mathtools, microtype}
\usepackage{hyperref}
	% https://tex.stackexchange.com/a/142276
	\pdfstringdefDisableCommands{\let\enspace\empty\let\noindent\empty}
	\hypersetup{colorlinks}
\usepackage{lmodern}

% increases the line spacing
\linespread{1.2}

%------------------------------------------------------------

%	modify with caution

\numberwithin{equation}{section}
\theoremstyle{plain}
	\newtheorem{thm}{Theorem}[section]
	\newtheorem{lem}[thm]{Lemma}
\theoremstyle{definition}
	\newtheorem{df}[thm]{Definition}
\theoremstyle{remark}
	\newtheorem{rem}[thm]{Remark}

%------------------------------------------------------------

%	examples of custom commands ("macros")

\newcommand{\bra}[1]{\left[#1\right]}
\newcommand{\pa}[1]{\left(#1\right)}

\begin{document}

\title{Title}
\author{Author}
\date{\today} % you can also enter the date manually here
\maketitle

\thispagestyle{empty}

%------------------------------------------------------------

%	\mainmatter

\section{Introduction}

\LaTeX{}, pronounced \emph{LAY-tek} or \emph{LAH-tek}, is a software system for typesetting documents.
It was created in the 1980s by the computer scientist Leslie Lamport, building on a more basic system called \TeX{}, which had been created in the 1970s by the computer scientist Donald Knuth.
Nowadays, these systems are extremely standard in scientific writing.

To create \LaTeX{} documents on your own computer, you need a \emph{compiler}, and for maximum convenience, an \emph{editor} as well.
You edit a \texttt{.tex} file in the editor, then click ``compile'' (or use a keyboard shortcut) to tell the compiler to create a \texttt{.pdf} document.
In doing so, the compiler will create a bunch of extra files with extensions like \texttt{.aux}, \texttt{.log}, and so on.
For this reason, it's helpful to keep all the documents for a given project within a single folder.

Alternatively, Overleaf (\url{https://www.overleaf.com/}) is a website where you can create \LaTeX{} documents online, after first creating a free account.
The makers also wrote a very nice guide to \LaTeX{} itself:
\begin{center}  \url{https://www.overleaf.com/learn/latex/Learn_LaTeX_in_30_minutes}
\end{center}
This webpage contains guidance on choosing and installing a compiler:
\begin{center}
\url{https://www.overleaf.com/learn/latex/Choosing_a_LaTeX_Compiler}
\end{center}
The links above were typeset using the \texttt{\textbackslash{}url\{ \}} command within a \texttt{center} environment.

\section{A New Section}

\subsection{Subsection}

An instance of a mathematical display:
\begin{align}\label{eq:1}
X = \bigcup_{i, j = 1}^\infty {(U_i \cap V_j)}.
\end{align}
Notice that \texttt{\textbackslash{}bigcap}, \texttt{\textbackslash{}bigcup} produce the larger symbols $\bigcap, \bigcup$, whereas \texttt{\textbackslash{}cap}, \texttt{\textbackslash{}cup} produce the smaller symbols $\cap, \cup$.

\subsection{Another Subsection} 

You can use commands like \texttt{\textbackslash{}textbf\{ \}} and  \texttt{\textbackslash{}textit\{ \}} to produce text in \textbf{bold} or \textit{italics}.
A shorter command for the latter is \texttt{\textbackslash{}emph\{ \}}.

\LaTeX{} offers several different alphabetic fonts in math mode, like
\begin{itemize} 
\item 	\texttt{\textbackslash{}mathbb\{ \}} for blackboard boldface ($\mathbb{A}$),

\item 	\texttt{\textbackslash{}mathbf\{ \}} for ordinary boldface ($\mathbf{A}$), 

\item 	\texttt{\textbackslash{}mathcal\{ \}} for calligraphic ($\mathcal{A}$),

\item 	\texttt{\textbackslash{}mathfrak\{ \}} for fraktur ($\mathfrak{A}$), and

\item 	\texttt{\textbackslash{}mathsf\{ \}} for sans-serif ($\mathsf{A}$).

\end{itemize}
Please use them wisely!

\LaTeX{} also offers several ways to decorate a symbol in math mode, like \texttt{\textbackslash{}bar\{ \}}, \texttt{\textbackslash{}hat\{ \}}, \texttt{\textbackslash{}vec\{ \}}, among others.
Use \texttt{\textbackslash{}widehat\{ \}} to get a hat that stretches:
Compare $\hat{X}$ and $\widehat{X}$.

\subsection{Yet Another Subsection} 

The list above was created using the \texttt{itemize} environment.
To get a numbered list, use the \texttt{enumerate} environment:
\begin{enumerate}
\item 	An item.

\item 	Another item.

\item 	A third item.

\end{enumerate}
Environments are also used to set apart blocks of text for theorems, proofs, and so on.

\begin{thm}[Terras]\label{thm:1}
A theorem environment.
\end{thm}

\begin{proof}
This \texttt{proof} environment contains an \texttt{align} environment:
\begin{align}
\frac{1}{2} \left[(x - y)^2 + (x + y)^2 \right]
	&= \frac{1}{2} \left[(x^2 - 2xy + y^2) + (x^2 + 2xy + y^2) \right]\\
	&= \frac{1}{2} [2x^2 + 2y^2]\\
	&= x^2 + y^2.
\end{align}
Notice that we used the commands \texttt{\textbackslash{}left} and \texttt{\textbackslash{}right} to make some brackets appropriately large.

To prevent \LaTeX{} from numbering each line, use \texttt{align*} instead of \texttt{align}.
To have a single number for the entire environment, put your math inside a \texttt{split} environment \emph{inside} the \texttt{align} environment.
\end{proof}

\begin{df}
A definition environment.
Here is a link to Theorem \ref{thm:1}.
\end{df}

\begin{rem}
A remark environment.
It contains a link to equation \eqref{eq:1}.
\end{rem}

\section{A Third Section}

Use \texttt{\textbackslash{}cite[ ]\{ \}} to create a citation.
Use an en dash (in \TeX: \texttt{-{}-}) instead of a hyphen (\texttt{-}) to typeset page ranges.
For instance,
\begin{center}
\texttt{\textbackslash{}cite[82--83]\{munkres\}}
produces
\cite[82--83]{munkres}.
\end{center}
The green text is a link to the bibliography entry below.

%------------------------------------------------------------

%	\backmatter
\frenchspacing
\begin{thebibliography}{11}
\bibitem[M]{munkres}
	J. Munkres.
	\textsl{Topology}.
	2nd Edition.
	Pearson Education, Ltd.\ (2014).
	
%	keep a blank space above \end{thebibliography}, in this TeX file
	
\end{thebibliography}

\end{document}
