%--------------------------------------------------------------------------

%	avoid modifying

\documentclass[a4paper,10pt,leqno]{amsart}
	\setlength{\textheight}{\paperheight}
	\addtolength{\textheight}{-2in}
	\calclayout
\RequirePackage[T1]{fontenc}
\RequirePackage[utf8]{inputenc}
\RequirePackage{xcolor}
\RequirePackage{amsfonts, amsmath, amssymb, amsthm}
%\RequirePackage[main=english,]{babel}
\RequirePackage{etoolbox, mathtools, microtype, stmaryrd, textcomp}
\RequirePackage{hyperref}
	%	http://tex.stackexchange.com/a/142276
	\pdfstringdefDisableCommands{\let\enspace\empty\let\noindent\empty}
	\hypersetup{colorlinks}
\RequirePackage{lmodern}

%	https://tex.stackexchange.com/a/47776
\makeatletter
\g@addto@macro \normalsize {%
 \setlength\abovedisplayskip{10pt plus 2pt minus 2pt}%
 \setlength\belowdisplayskip{10pt plus 2pt minus 2pt}%
}
\makeatother

\linespread{1.25}

%--------------------------------------------------------------------------

%	modify with caution

	\numberwithin{equation}{section}
	\theoremstyle{plain}
		\newtheorem{thm}{Theorem}[section]
	\theoremstyle{plain}
		\newtheorem{conj}[thm]{Conjecture}
		\newtheorem{cor}[thm]{Corollary}
		\newtheorem{lem}[thm]{Lemma}
		\newtheorem{prop}[thm]{Proposition}
	\theoremstyle{definition}
		\newtheorem{df}[thm]{Definition}
		\newtheorem{ex}[thm]{Example}
		\newtheorem{xc}[thm]{Exercise}
	\theoremstyle{remark}
		\newtheorem{rem}[thm]{Remark}

%--------------------------------------------------------------------------

%	http://tex.stackexchange.com/a/5776/20882
\newcommand{\define}[3]{\expandafter#1\csname#3\endcsname{#2{#3}}}
\forcsvlist{\define{\DeclareMathOperator}{}}{ad,codim,coker,gr,id,im,lcm,ord,pr,rk,sgn,supp,tr}
\forcsvlist{\define{\DeclareMathOperator}{}}{Ad,Aff,Alt,Aut,End,Gal,Heis,Hom,Ind,Inf,Irr,Mat,Sym}
\forcsvlist{\define{\DeclareMathOperator}{}}{GL,PGL,PSL,SL,SO,SU}
\forcsvlist{\define{\newcommand}{\mathrm}}{ab,op,sep,tor}

\newcommand{\inject}{\hookrightarrow}
\newcommand{\normal}{\vartriangleleft}
\newcommand{\normaleq}{\trianglelefteq}
\newcommand{\surject}{\twoheadrightarrow}
\newcommand{\bra}[1]{\left[#1\right]}
\newcommand{\pa}[1]{\left(#1\right)}
\newcommand{\mat}[1]{\begin{matrix}#1\end{matrix}}
\newcommand{\pmat}[1]{\pa{\mat{#1}}}
\newcommand{\smat}[1]{\begin{smallmatrix}#1\end{smallmatrix}}
\newcommand{\psmat}[1]{\pa{\smat{#1}}}

\begin{document}

\title{Title}
\author{Author}
%\address{Massachusetts Institute of Technology, 77 Massachusetts Avenue, Cambridge, MA 02139}
\maketitle

\thispagestyle{empty}

%--------------------------------------------------------------------------

%	\mainmatter

\section{Section}

\subsection{Subsection}

A formula that could come in handy:
\begin{align}\label{eq:1}
\hat{f}(\chi) = \sum_{g \in G} f(g)\chi(g^{-1}).
\end{align}
We can use \texttt{\textbackslash{}widehat} to get a hat that stretches: for instance, $\widehat{F}$.

\begin{thm}[Terras]\label{thm:1}
A theorem environment.
\end{thm}

\begin{df}
A definition environment.
It contains a link to Theorem \ref{thm:1}.
\end{df}

\begin{rem}
A remark environment.
It contains a link to equation \eqref{eq:1}.
\end{rem}

Here are examples of the matrix commands.
To typeset a matrix inline with the text, use \texttt{\textbackslash{}psmat\{ \}} like so: $\psmat{a&b\\c&d}$.
To typeset it as a display, use \texttt{\textbackslash{}pmat\{ \}}:
\begin{align}
\pmat{1&x&z\\ &1&y \\ &&1}.
\end{align}
If the matrix is very large, then display will look better than inline.

Here is a sequence of equations formatted with the \texttt{align} environment:
\begin{align}
\begin{split}
\left\| T^m p - u \right\|_2^2
&=	\left\|\pa{\sum_{j = 1}^n \langle p, \phi_j\rangle \lambda_j^m \phi_j} - u\right\|_2^2\\
&=	\left\|\sum_{j = 2}^n \langle p, \phi_j\rangle \lambda_j^m \phi_j\right\|_2^2\\
&=	\sum_{j = 2}^n |\langle p, \phi_j\rangle|^2 |\lambda_j|^{2m}.
\end{split}\end{align}
I used the \texttt{split} environment to avoid numbering each line separately.

Lastly:
The AMS helpfully provides a \href{https://mathscinet.ams.org/msnhtml/serials.pdf}{list} of the correct abbreviations to use in citations for the names of journals.
Use an en dash (in \TeX: \texttt{-{}-}) instead of a hyphen (\texttt{-}) to typeset page ranges.

%--------------------------------------------------------------------------

%	\backmatter
\frenchspacing
\begin{thebibliography}{11}
\bibitem[T]{terras}
	A. Terras.
	\textsl{Fourier Analysis on Finite Groups and Applications}.
	Cambridge University Press (1999).
	
\bibitem[W]{wiles}
	A. Wiles.
	Modular Elliptic Curves and Fermat's Last Theorem.
	\textsl{Ann. of Math. (2)},
	\textbf{141}(93)
	(May, 1995),
	443--551.

%	space needed here
	
\end{thebibliography}

\end{document}
